\documentclass[11pt]{article}

\topmargin -.5in
\textheight 9in
\oddsidemargin -.25in
\evensidemargin -.25in
\textwidth 7in

\usepackage{amsthm}
\usepackage{amssymb, amsmath}
\usepackage{fancyhdr}

\pagestyle{fancy}
\fancyhf{}
\lhead{2025 Fall}
\rhead{CS-101 HW1}
\cfoot{\thepage}
\rfoot{Your Name}

\newcommand{\numpy}{{\tt numpy}}    % tt font for numpy
\newcommand{\floor}[1]{\left\lfloor #1 \right\rfloor}
\newcommand{\ceil}[1]{\left\lceil #1 \right\rceil}

\newtheorem{theorem}{Theorem}[section]
\newtheorem{lemma}[theorem]{Lemma}
\renewcommand\qedsymbol{$\blacksquare$}

\begin{document}

\author{Your Name (Student ID)}
\title{CS-101 Homework 1}
\maketitle

\medskip

\begin{enumerate}

\item
\textbf{Problem 1. [Think through]}

\begin{proof}[Think.]

% ========== Just examples, please delete before submitting
Use inline equations for simple math $1+1=2$, and centered equations for more involved or important equations
\begin{equation}
    a^2 + b^2 = c^2.
\end{equation}

Some people like to write scalars without boldface $x+y=1$ and vectors or matrices in boldface
\begin{equation}
    \mathbf{A} \mathbf{x} = \mathbf{b}.
\end{equation}

An example of a matrix \LaTeX:
\begin{equation}
    \mathbf{A} = \left(
    \begin{array}{ccc}
    3 & -1 & 2 \\ 	
    0 & 1 & 2 \\ 
    1 & 0 & -1 \\
\end{array} 
\right).  
\end{equation}

With a labeled equation such as the following:
\begin{equation}
    \label{accel}
    \frac{d^2 x}{d t^2} = a
\end{equation}
you can referrer to the equation later. In equation \ref{accel} we defined acceleration.
% ========== END examples

\end{proof}

\item
\textbf{Problem 2.}

\begin{proof}

% ========== Just examples, please delete before submitting
Use inline equations for simple math $1+1=2$, and centered equations for more involved or important equations
\begin{equation}
    a^2 + b^2 = c^2.
\end{equation}

Some people like to write scalars without boldface $x+y=1$ and vectors or matrices in boldface
\begin{equation}
    \mathbf{A} \mathbf{x} = \mathbf{b}.
\end{equation}

An example of a matrix \LaTeX:
\begin{equation}
    \mathbf{A} = \left(
    \begin{array}{ccc}
    3 & -1 & 2 \\ 	
    0 & 1 & 2 \\ 
    1 & 0 & -1 \\
\end{array} 
\right).  
\end{equation}

With a labeled equation such as the following:
\begin{equation}
    \label{accel}
    \frac{d^2 x}{d t^2} = a
\end{equation}
you can referrer to the equation later. In equation \ref{accel} we defined acceleration.
% ========== END examples

\end{proof}



\end{enumerate}

\end{document}
